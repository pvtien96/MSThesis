\chapter{Tables}
\begin{table}
	\label{tab:fastConfig}
	\begin{tabular}{|l|l|}
		\hline
		Key            & Value                                           \\ \hline
		\_BASE\_:      & "../Base-RCNN-FPN.yaml"                         \\ \hline
		MODEL:         &                                                 \\ \hline
		WEIGHTS:       & "detectron2://ImageNetPretrained/MSRA/R-50.pkl" \\ \hline
		MASK\_ON       & False                                           \\ \hline
		RESNETS:       &                                                 \\ \hline
		DEPTH:         & 50                                              \\ \hline
		OUT\_FEATURES: & {[}"res2", "res3", "res4",   "res5"{]}          \\ \hline
		FPN:           &                                                 \\ \hline
		IN\_FEATURES:  & {[}"res2", "res3", "res4",   "res5"{]}          \\ \hline
		SOLVER:        & \multirow{2}{*}{(210000, 250000)}               \\ \cline{1-1}
		STEPS:         &                                                 \\ \hline
		MAX\_ITER:     & 270000                                          \\ \hline
		ROI\_HEADS:    & \multirow{2}{*}{"StandardROIHeads"}             \\ \cline{1-1}
		NAME:          &                                                 \\ \hline
		IN\_FEATURES:  & {[}"p2", "p3", "p4",   "p5"{]}                  \\ \hline
	\end{tabular}
	\caption{FasterRCNN\_R\_50\_FPN3x training configuration file. Detail information field is explained at the Detectron2's application programming interface (API) documentation. The main difference of FasterRCNN and MaskRCNN in term of configuration is MaskRCNN's option MASK\_ON value.}
\end{table}
\begin{table}
	\label{tab:yoloConfig}
	\begin{tabular}{|l|l|}
		\hline
		Key              & Value                                                                                                                                                                                                                                                                                                                                                                                                                                                                                                                                                                                                                                                                                                                                                                                                                                                                                                                                                                           \\ \hline
		nc:              & 1 \# number of classes                                                                                                                                                                                                                                                                                                                                                                                                                                                                                                                                                                                                                                                                                                                                                                                                                                                                                                                                                          \\ \hline
		depth\_multiple: & 0.33 \# model depth multiple                                                                                                                                                                                                                                                                                                                                                                                                                                                                                                                                                                                                                                                                                                                                                                                                                                                                                                                                                    \\ \hline
		width\_multiple: & 0.50 \# layer channel multiple                                                                                                                                                                                                                                                                                                                                                                                                                                                                                                                                                                                                                                                                                                                                                                                                                                                                                                                                                  \\ \hline
		anchors:         & \begin{tabular}[c]{@{}l@{}}- {[}10,13, 16,30, 33,23{]}  \# P3/8\\ - {[}30,61, 62,45, 59,119{]}  \# P4/16\\ - {[}116,90, 156,198, 373,326{]}  \# P5/32\end{tabular}                                                                          \\ \hline
		backbone:        & \begin{tabular}[c]{@{}l@{}}\# {[}from, number, module, args{]}\\   {[}{[}-1, 1, Focus, {[}64, 3{]}{]},  \# 0-P1/2\\    {[}-1, 1, Conv, {[}128, 3, 2{]}{]},  \# 1-P2/4\\    {[}-1, 3, BottleneckCSP, {[}128{]}{]},\\    {[}-1, 1, Conv, {[}256, 3, 2{]}{]},  \# 3-P3/8\\    {[}-1, 9, BottleneckCSP, {[}256{]}{]},\\    {[}-1, 1, Conv, {[}512, 3, 2{]}{]},  \# 5-P4/16\\    {[}-1, 9, BottleneckCSP, {[}512{]}{]},\\    {[}-1, 1, Conv, {[}1024, 3, 2{]}{]},  \# 7-P5/32\\    {[}-1, 1, SPP, {[}1024, {[}5, 9, 13{]}{]}{]},\\    {[}-1, 3, BottleneckCSP, {[}1024, False{]}{]},  \# 9\\   {]}\end{tabular}                                                                                                                                                                                                                                                                                                                                                                      \\ \hline
		head:            & \begin{tabular}[c]{@{}l@{}}{[}{[}-1, 1, Conv, {[}512, 1, 1{]}{]},\\    {[}-1, 1, nn.Upsample, {[}None, 2, 'nearest'{]}{]},\\    {[}{[}-1, 6{]}, 1, Concat, {[}1{]}{]},  \# cat backbone P4\\    {[}-1, 3, BottleneckCSP, {[}512, False{]}{]},  \# 13\\  \\    {[}-1, 1, Conv, {[}256, 1, 1{]}{]},\\    {[}-1, 1, nn.Upsample, {[}None, 2, 'nearest'{]}{]},\\    {[}{[}-1, 4{]}, 1, Concat, {[}1{]}{]},  \# cat backbone P3\\    {[}-1, 3, BottleneckCSP, {[}256, False{]}{]},  \# 17 (P3/8-small)\\  \\    {[}-1, 1, Conv, {[}256, 3, 2{]}{]},\\    {[}{[}-1, 14{]}, 1, Concat, {[}1{]}{]},  \# cat head P4\\    {[}-1, 3, BottleneckCSP, {[}512, False{]}{]},  \# 20  (P4/16-medium)\\  \\    {[}-1, 1, Conv, {[}512, 3, 2{]}{]},\\    {[}{[}-1, 10{]}, 1, Concat, {[}1{]}{]},  \# cat head P5\\    {[}-1, 3, BottleneckCSP, {[}1024, False{]}{]},  \# 23  (P5/32-large)\\  \\    {[}{[}17, 20, 23{]}, 1, Detect, {[}nc, anchors{]}{]},  \# Detect(P3, P4, P5){]}\end{tabular} \\ \hline
	\end{tabular}
	\caption{YOLOv4x training configuration file. Detail information field is explained at Ultralytic API.}
\end{table}
\begin{table}
	\label{tab:d2custom}
	\begin{tabular}{|l|l|}
		\hline
		Field         & Meaning                                                                                                                                                                                                                                         \\ \hline
		file\_name    & \begin{tabular}[c]{@{}l@{}}the full path to the image file. Rotation or flipping may   be applied if the\\ image has EXIF metadata.\end{tabular}                                                                                                \\ \hline
		height, width & integer.   The shape of the image.                                                                                                                                                                                                              \\ \hline
		image\_id     & \begin{tabular}[c]{@{}l@{}}(str   or int): a unique id that identifies this image. Required by many\\ evaluators  to identify the images, but a dataset may use it for different\\ purposes.\end{tabular}                                       \\ \hline
		annotations   & \begin{tabular}[c]{@{}l@{}}(list{[}dict{]}): Required by instance detection, segmentation or keypoint\\ detection tasks. Each dict corresponds to annotations of one instance in\\ this image.\end{tabular}                                     \\ \hline
		bbox          & \begin{tabular}[c]{@{}l@{}}(list{[}float{]},   required): list of 4 numbers representing the bounding box of\\ the instance.\end{tabular}                                                                                                       \\ \hline
		bbox\_mode    & \begin{tabular}[c]{@{}l@{}}(int, required): the format of bbox. It must be a member\\  of structures.BoxMode. Currently supports: \\ BoxMode.XYXY\_ABS, BoxMode.XYWH\_ABS.\end{tabular}                                                         \\ \hline
		category\_id  & \begin{tabular}[c]{@{}l@{}}(int,   required): an integer in the range {[}0, num\_categories-1{]} representing\\ the category label. The value num\_categories is reserved to represent the\\ “background” category, if applicable.\end{tabular} \\ \hline
		segmentation  & (list{[}list{[}float{]}{]}   or dict): the segmentation mask of the instance.                                                                                                                                                                   \\ \hline
	\end{tabular}
	\caption{Detectron2’s custom dataset format.}
\end{table}
%\clearpage
%\newpage

