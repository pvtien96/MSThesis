% -*-latex-*-
% 
% For questions, comments, concerns or complaints:
% thesis@mit.edu
% 
%
% $Log: cover.tex,v $
% Revision 1.9  2019/08/06 14:18:15  cmalin
% Replaced sample content with non-specific text.
%
% Revision 1.8  2008/05/13 15:02:15  jdreed
% Degree month is June, not May.  Added note about prevdegrees.
% Arthur Smith's title updated
%
% Revision 1.7  2001/02/08 18:53:16  boojum
% changed some \newpages to \cleardoublepages
%
% Revision 1.6  1999/10/21 14:49:31  boojum
% changed comment referring to documentstyle
%
% Revision 1.5  1999/10/21 14:39:04  boojum
% *** empty log message ***
%
% Revision 1.4  1997/04/18  17:54:10  othomas
% added page numbers on abstract and cover, and made 1 abstract
% page the default rather than 2.  (anne hunter tells me this
% is the new institute standard.)
%
% Revision 1.4  1997/04/18  17:54:10  othomas
% added page numbers on abstract and cover, and made 1 abstract
% page the default rather than 2.  (anne hunter tells me this
% is the new institute standard.)
%
% Revision 1.3  93/05/17  17:06:29  starflt
% Added acknowledgements section (suggested by tompalka)
% 
% Revision 1.2  92/04/22  13:13:13  epeisach
% Fixes for 1991 course 6 requirements
% Phrase "and to grant others the right to do so" has been added to 
% permission clause
% Second copy of abstract is not counted as separate pages so numbering works
% out
% 
% Revision 1.1  92/04/22  13:08:20  epeisach

% NOTE:
% These templates make an effort to conform to the MIT Thesis specifications,
% however the specifications can change. We recommend that you verify the
% layout of your title page with your thesis advisor and/or the MIT 
% Libraries before printing your final copy.
\title{Hand detection, segmentation and tracking from egocentric vision}

\author{Van-Tien Pham}
% If you wish to list your previous degrees on the cover page, use the 
% previous degrees command:
%       \prevdegrees{A.A., Harvard University (1985)}
% You can use the \\ command to list multiple previous degrees
%       \prevdegrees{B.S., University of California (1978) \\
%                    S.M., Massachusetts Institute of Technology (1981)}
\department{School of Information Technology and Communication}

% If the thesis is for two degrees simultaneously, list them both
% separated by \and like this:
% \degree{Doctor of Philosophy \and Master of Science}
\degree{Master of Science in Information System and Communication}

% As of the 2007-08 academic year, valid degree months are September, 
% February, or June.  The default is June.
\degreemonth{October}
\degreeyear{2020}
\thesisdate{October 10, 2020}

%% By default, the thesis will be copyrighted to MIT.  If you need to copyright
%% the thesis to yourself, just specify the `vi' documentclass option.  If for
%% some reason you want to exactly specify the copyright notice text, you can
%% use the \copyrightnoticetext command.  
%\copyrightnoticetext{\copyright IBM, 1990.  Do not open till Xmas.}

% If there is more than one supervisor, use the \supervisor command
% once for each.
\supervisor{Thi-Thanh-Hai Tran}{Associate Professor}

% This is the department committee chairman, not the thesis committee
% chairman.  You should replace this with your Department's Committee
% Chairman.
\chairman{Chairman}{Chairman, Department Committee on Graduate Theses}

% Make the titlepage based on the above information.  If you need
% something special and can't use the standard form, you can specify
% the exact text of the titlepage yourself.  Put it in a titlepage
% environment and leave blank lines where you want vertical space.
% The spaces will be adjusted to fill the entire page.  The dotted
% lines for the signatures are made with the \signature command.
\maketitle

% The abstractpage environment sets up everything on the page except
% the text itself.  The title and other header material are put at the
% top of the page, and the supervisors are listed at the bottom.  A
% new page is begun both before and after.  Of course, an abstract may
% be more than one page itself.  If you need more control over the
% format of the page, you can use the abstract environment, which puts
% the word "Abstract" at the beginning and single spaces its text.

%% You can either \input (*not* \include) your abstract file, or you can put
%% the text of the abstract directly between the \begin{abstractpage} and
%% \end{abstractpage} commands.

% First copy: start a new page, and save the page number.
\cleardoublepage
% Uncomment the next line if you do NOT want a page number on your
% abstract and acknowledgments pages.
\pagestyle{empty}
\setcounter{savepage}{\thepage}
\begin{abstractpage}
% $Log: abstract.tex,v $
% Revision 1.1  93/05/14  14:56:25  starflt
% Initial revision
% 
% Revision 1.1  90/05/04  10:41:01  lwvanels
% Initial revision
% 
%
%% The text of your abstract and nothing else (other than comments) goes here.
%% It will be single-spaced and the rest of the text that is supposed to go on
%% the abstract page will be generated by the abstractpage environment.  This
%% file should be \input (not \include 'd) from cover.tex.
Multiple object tracking is the process of assigning unique and consistent identities to objects throughout a video sequence. A de-facto approach to multiple object tracking, and object tracking in general, is to use a method called tracking by detection. Tracking by detection is a two-stage procedure: an object detection or segmentation algorithm fist detects objects in a given frame, these detected objects are then associated with already tracked objects in a second step by a tracking algorithm. Egocentric vision is an emerging field of computer vision that is characterized by the acquisition of images and video from the first-person perspective. In egocentric view, the two human hands are essential in the execution of actions and characterizing their movements and trajectories are the principal cues to define and recognize actions.
\\One of the main concerns of this thesis is to develop an automatic tracking by detection algorithm that extracts hands positions and identities in consequence frames from egocentric surveillance video. The proposed framework consists of state-of-the-art detectors from RCNN and YOLO family models combined with the SORT or DeepSORT for object tracking task. The thesis aims to explore how the stand-alone alone performance of the object detection algorithm correlates with overall performance of a tracking-by-detection system. Finally, the thesis investigates how the use of visual descriptors of DeepSORT in the tracking stage of a tracking-by-detection system effects performance.
\\Results presented in this thesis suggest that the capacity of the object detection algorithm is highly indicative of the overall performance of the tracking-by detection system. Further, this thesis also shows how the use of visual descriptors in the tracking stage can reduce the number of identity switches and thereby increase performance of the whole system. This thesis also presents a new egocentric hand tracking dataset Micand32 for future researches.


\end{abstractpage}

% Additional copy: start a new page, and reset the page number.  This way,
% the second copy of the abstract is not counted as separate pages.
% Uncomment the next 6 lines if you need two copies of the abstract
% page.
% \setcounter{page}{\thesavepage}
% \begin{abstractpage}
% % $Log: abstract.tex,v $
% Revision 1.1  93/05/14  14:56:25  starflt
% Initial revision
% 
% Revision 1.1  90/05/04  10:41:01  lwvanels
% Initial revision
% 
%
%% The text of your abstract and nothing else (other than comments) goes here.
%% It will be single-spaced and the rest of the text that is supposed to go on
%% the abstract page will be generated by the abstractpage environment.  This
%% file should be \input (not \include 'd) from cover.tex.
Multiple object tracking is the process of assigning unique and consistent identities to objects throughout a video sequence. A de-facto approach to multiple object tracking, and object tracking in general, is to use a method called tracking by detection. Tracking by detection is a two-stage procedure: an object detection or segmentation algorithm fist detects objects in a given frame, these detected objects are then associated with already tracked objects in a second step by a tracking algorithm. Egocentric vision is an emerging field of computer vision that is characterized by the acquisition of images and video from the first-person perspective. In egocentric view, the two human hands are essential in the execution of actions and characterizing their movements and trajectories are the principal cues to define and recognize actions.
\\One of the main concerns of this thesis is to develop an automatic tracking by detection algorithm that extracts hands positions and identities in consequence frames from egocentric surveillance video. The proposed framework consists of state-of-the-art detectors from RCNN and YOLO family models combined with the SORT or DeepSORT for object tracking task. The thesis aims to explore how the stand-alone alone performance of the object detection algorithm correlates with overall performance of a tracking-by-detection system. Finally, the thesis investigates how the use of visual descriptors of DeepSORT in the tracking stage of a tracking-by-detection system effects performance.
\\Results presented in this thesis suggest that the capacity of the object detection algorithm is highly indicative of the overall performance of the tracking-by detection system. Further, this thesis also shows how the use of visual descriptors in the tracking stage can reduce the number of identity switches and thereby increase performance of the whole system. This thesis also presents a new egocentric hand tracking dataset Micand32 for future researches.


% \end{abstractpage}

\cleardoublepage

\section*{Acknowledgments}

First of all, I might want to offer my special thanks to my supervisor, Assoc. Prof. Tran Thi Thanh Hai. I'd really appreciate everything she've guided me all through this thesis.

Many thanks to my colleagues at Viettel High Technology and Industry Coporation for supporting me in engineering technique. Also, I might want to express gratitude toward Assoc. Prof. Vu Hai and alumni at MICA Institute, Hanoi University of Science and Technology  for giving me significant suggestions.

Deep inside my heart, I wish to show my gratefulness to my parent and my sister for always inspiring and trusting me in every of my steps.

%%%%%%%%%%%%%%%%%%%%%%%%%%%%%%%%%%%%%%%%%%%%%%%%%%%%%%%%%%%%%%%%%%%%%%
% -*-latex-*-

