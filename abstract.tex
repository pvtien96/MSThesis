% $Log: abstract.tex,v $
% Revision 1.1  93/05/14  14:56:25  starflt
% Initial revision
% 
% Revision 1.1  90/05/04  10:41:01  lwvanels
% Initial revision
% 
%
%% The text of your abstract and nothing else (other than comments) goes here.
%% It will be single-spaced and the rest of the text that is supposed to go on
%% the abstract page will be generated by the abstractpage environment.  This
%% file should be \input (not \include 'd) from cover.tex.
%Multiple object tracking is the process of assigning unique and consistent identities to objects throughout a video sequence. A de-facto approach to multiple object tracking, and object tracking in general, is to use a method called tracking by detection. Tracking by detection is a two-stage procedure: an object detection or segmentation algorithm fist detects objects in a given frame, these objects are then associated with already tracked objects by a tracking algorithm. 
Egocentric vision is an emerging field of computer vision that is characterized by the acquisition of images and video from the first-person perspective. In egocentric view, the two human hands are essential in the execution of actions and obtaining their movements and trajectories are the principal cues to define and recognize actions.
One of the main concerns of this thesis is to build an automatic tracking by detection software that extract hands positions and identities in consequence frames from egocentric surveillance video supplied by the project “Understanding Human Daily Life Activities from Egocentric Vision Using Advanced Technologies of Deep Learning” sponsored by NAFOSTED. The framework consists of many state-of-the-art detectors from RCNN and YOLO family models combined with the SORT or DeepSORT. The thesis then goes on to explore how the stand-alone alone performance of the object detection algorithm correlates with overall performance of a tracking-by-detection system. Finally, the thesis investigates how the use of visual descriptors of DeepSORT in the tracking stage of a tracking-by-detection system effects performance.
Results presented in this thesis suggest that the capacity of the object detection algorithm is highly indicative of the overall performance of the tracking-by detection system. Further, this thesis also shows how the use of visual descriptors in the tracking stage can reduce the number of identity switches and thereby increase performance of the whole system. This thesis also presents a new egocentric hand tracking dataset MICAND32 for future researches.

